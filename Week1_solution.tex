\documentclass[a4paper]{article}
\usepackage{xeCJK}
\usepackage{indentfirst}
\usepackage{amsmath,amssymb,amsthm}
\usepackage{rotating}
\usepackage[margin=2.5cm]{geometry}
\usepackage{booktabs}
\theoremstyle{definition}
\newenvironment{exercise}[1][0.0.0]{\noindent\textbf{习题 #1}\hspace{1em}}{\quad\\}
\newenvironment{solution}[1][]{\noindent\textbf{\textit{解:}}}{\quad\\}
\newenvironment{case}[2]{\noindent\textbf{情况 #1:}#2\\}{\quad\\}
\newenvironment{step}[2]{\noindent\textbf{步骤 #1:}#2\\}{\quad\\}
\linespread{1.4}

\author{刘思齐老师;廖汶锋、乐永琪、杜若甫助教}

\title{\textbf{线性代数——第一周作业解答}}

\begin{document}
	\maketitle
	
	\iffalse week 1
	练习 0.2.2
	练习 0.3.1
	练习 0.3.4
	练习 0.3.9
	练习 1.1.2
	练习 1.1.5
	练习 1.1.8
	\fi
	
	\begin{exercise}[0.2.2]
		经典逻辑的基本公理之一是排中律,即对任意命题 $A$,$A$ 不能既不真又不假。反证法可以看作排中律的推论,即如果我们发现 $A$ 不能是错的,那么 $A$ 就只能是对的。因此,想要证明命题 $A$ 真,不妨挑一个命题 $B$,先证明 $B\Rightarrow A$,再证明 $(\neg B)\Rightarrow A$.那么 $A$ 就必须是真的了。给定命题:存在两个无理数 $a$、$b$,满足 $a^{b}$ 是一个有理数。请先假设 $\sqrt{2}^{\sqrt{2}}$ 是有理数,再假设 $\sqrt{2}^{\sqrt{2}}$ 是无理数,对两种情形讨论来证明这个命题。
	\end{exercise}
	
	\begin{solution}
		取命题 $A$ 为“存在两个无理数 $a$、$b$,满足 $a^{b}$ 是一个有理数”,而命题 $B$ 为 “$\sqrt{2}^{\sqrt{2}}$ 是有理数” 。
		
		\begin{case}{1}{证明 $B\Rightarrow A$}
			取$a=b=\sqrt{2}$,则推导出命题 $A$ 成立;
		\end{case}
		\begin{case}{2}{证明 $(\neg B)\Rightarrow A$}
			取$a=\sqrt{2}^{\sqrt{2}}$、$b=\sqrt{2}$,则推导出$a^{b}=\sqrt{2}^{\sqrt{2}\times\sqrt{2}}=2\in\mathbb{Q}$,即命题 $A$ 成立。
		\end{case}
		所以存在两个无理数 $a$、$b$,满足 $a^{b}$ 是一个有理数。
	\end{solution}
	
	\begin{exercise}[0.3.1]
		判断下列映射是否为单射、满射、双射,并写出双射的逆映射:
		\begin{table}[htbp]
			\centering
			\begin{tabular}{p{0.3\textwidth}p{0.3\textwidth}p{0.3\textwidth}}
				1. $f: \mathbb{R}\rightarrow\mathbb{R}$&2. $f: \mathbb{R}\rightarrow\mathbb{R}$&3. $f: \mathbb{R}\rightarrow\mathbb{R}$\\
				\hspace{3em}$x\mapsto x+1$&\hspace{3em}$x\mapsto 2x$&\hspace{3em}$x\mapsto 3$\\
				4. $f: \mathbb{R}\rightarrow\mathbb{R}$&5. $f: (-\infty,0]\rightarrow\mathbb{R}$&6. $f: \mathbb{R}\rightarrow(0,+\infty)$\\
				\hspace{3em}$x\mapsto x^2$&\hspace{3em}$x\mapsto x^2$&\hspace{3em}$x\mapsto e^{x}$\\
				7. $f: \left[-\frac{3\pi}{2},-\frac{\pi}{2}\right]\rightarrow[-1,1]$&&\\
				\hspace{3em}$x\mapsto\sin{x}$&&
			\end{tabular}
		\end{table}
	\end{exercise}
	\begin{solution}
		\begin{table}[htbp]
			\centering
			\begin{tabular}{p{0.05\textwidth}p{0.05\textwidth}p{0.05\textwidth}p{0.05\textwidth}p{0.5\textwidth}}
				\toprule
				序号&单射&满射&双射&双射的逆映射\\
				\midrule
				1.&$\surd$&$\surd$&$\surd$&$f^{-1}: \mathbb{R}\rightarrow\mathbb{R},\ x\mapsto x-1$\\
				2.&$\surd$&$\surd$&$\surd$&$f^{-1}: \mathbb{R}\rightarrow\mathbb{R},\ x\mapsto x/2$\\
				3.&$\times$&$\times$&$\times$&$\times$\\
				4.&$\times$&$\times$&$\times$&$\times$\\
				5.&$\surd$&$\times$&$\times$&$\times$\\
				6.&$\surd$&$\surd$&$\surd$&$f^{-1}: (0,+\infty)\rightarrow\mathbb{R},\ x\mapsto \ln{x}$\\
				7.&$\surd$&$\surd$&$\surd$&$f^{-1}:  [-1,1]\rightarrow\left[-\frac{3\pi}{2},-\frac{\pi}{2}\right],\ x\mapsto -(\pi+\arcsin{x})$\\
				\bottomrule
			\end{tabular}
		\end{table}
	\end{solution}
	
	\begin{exercise}[0.3.4]
		下列 $\mathbb{R}$ 上的变换,哪些满足交换律 $f\circ g=g\circ f$?
		\begin{table}[htbp]
			\centering
			\begin{tabular}{p{0.5\textwidth}p{0.5\textwidth}}
				1. $f(x)=x+1$,$g(x)=2x$&2. $f(x)=x^{2}$,$g(x)=x^{3}$\\
				3. $f(x)=2^{x}$,$g(x)=3^{x}$&4. $f(x)=2x+1$,$g(x)=3x+2$\\
				5. $f(x)=2x+1$,$g(x)=3x+1$&6. $f(x)=\sin{x}$,$g(x)=\cos{x}$
			\end{tabular}
		\end{table}
	\end{exercise}
	\begin{solution}
		\begin{table}[htbp]
			\centering
			\begin{tabular}{p{0.05\textwidth}p{0.1\textwidth}p{0.1\textwidth}p{0.15\textwidth}}
				\toprule
				序号&$f\circ g$&$g\circ f$&满足交换律?\\
				\midrule
				1.&$2x+1$&$2x+2$&$\times$\\
				2.&$x^{6}$&$x^{6}$&$\surd$\\
				3.&$2^{3^{x}}$&$3^{2^{x}}$&$\times$\\
				4.&$6x+5$&$6x+5$&$\surd$\\
				5.&$6x+3$&$6x+4$&$\times$\\
				6.&$\sin(\cos{x})$&$\cos(\sin{x})$&$\times$\\
				\bottomrule
			\end{tabular}
		\end{table}
	\end{solution}
	
	\begin{exercise}[0.3.9]
		给定映射$h$、$g$ 和 $f=g\circ h$,证明,若 $f$ 是双射,则 $h$ 是单射,$g$ 是满射。
	\end{exercise}
	
	\begin{solution}
		记 $h : X\rightarrow Y$、$g : Y\rightarrow Z$,则有 $f : X\rightarrow Z$。
		
		\begin{step}{1}{证明:$h$ 是单射}
			因为 $f$ 是单射,所以对于任意 $x_1$、$x_2\in X$ 且 $x_1\neq x_2$ 来说,都有 $g\circ h(x_1) \neq g\circ h(x_2)$。\\另一方面,在映射 $g$ 之下,$g\circ h(x_1)$ 和 $g\circ h(x_2)$ 的原像分别为 $h(x_1)$ 和 $h(x_2)$,所以必定有 $h(x_1) \neq h(x_2)$,从而推导出 $h$ 是单射。
		\end{step}
		\begin{step}{2}{证明:$g$ 是满射}
			因为 $f$ 是满射,所以对于任意 $z\in Z$,都存在$x\in X$,使得$f(x)=z$。\\
			同时,因为$z=f(x)=g(h(x))$ 且 $h(x)\in Y$,所以 $g$ 是满射。
		\end{step}
	\end{solution}
	
	\begin{exercise}[1.1.2]
		如果平面上的向量 $\begin{bmatrix}
			a\\b
		\end{bmatrix}$ 与 $\begin{bmatrix}
			c\\d
		\end{bmatrix}$ 共线,那么 $\begin{bmatrix}
			a\\c
		\end{bmatrix}$ 与 $\begin{bmatrix}
			b\\d
		\end{bmatrix}$ 是否共线 ?
	\end{exercise}
	\begin{solution}
		\begin{case}{1}{$\begin{bmatrix}
					a\\b
				\end{bmatrix}\neq \textbf{0}$}
			此时,必然存在一个实数 $\lambda$ 满足
			\[\begin{bmatrix}
				c\\d
			\end{bmatrix}=\lambda\begin{bmatrix}
				a\\b
			\end{bmatrix}\]
			所以有
			\[\begin{bmatrix}
				a\\c
			\end{bmatrix}=a\begin{bmatrix}
				1\\\lambda
			\end{bmatrix},\hspace{1em}\begin{bmatrix}
				b\\d
			\end{bmatrix}=b\begin{bmatrix}
				1\\\lambda
			\end{bmatrix}\]
			因此,$\begin{bmatrix}
				a\\c
			\end{bmatrix}$、$\begin{bmatrix}
				b\\d
			\end{bmatrix}$ 与 $\begin{bmatrix}
				1\\\lambda
			\end{bmatrix}$ 三者共线。
		\end{case}\newpage
		\begin{case}{2}{$\begin{bmatrix}
					a\\b
				\end{bmatrix} = \textbf{0}$}
			显然 $\begin{bmatrix}
				a\\c
			\end{bmatrix}$、$\begin{bmatrix}
				b\\d
			\end{bmatrix}$ 与 $\begin{bmatrix}
				0\\1
			\end{bmatrix}$ 三者共线。
		\end{case}
	\end{solution}
	
	\begin{exercise}[1.1.5]
		判断下列映射是否为线性映射:
		\begin{table}[htbp]
			\centering
			\begin{tabular}{p{0.3\textwidth}p{0.3\textwidth}p{0.3\textwidth}}
				1. $f: \mathbb{R}\rightarrow\mathbb{R}$&2. $f: \mathbb{R}\rightarrow\mathbb{R}$&3. $f: \mathbb{R}\rightarrow\mathbb{R}$\\
				\hspace{3em}$x\mapsto x+1$&\hspace{3em}$x\mapsto 2x$&\hspace{3em}$x\mapsto 0$\\
				4. $f: \mathbb{R}\rightarrow\mathbb{R}$&5. $f: \mathbb{R}\rightarrow\mathbb{R}$&6. $f: \mathbb{R}\rightarrow\mathbb{R}$\\
				\hspace{3em}$x\mapsto 1$&\hspace{3em}$x\mapsto x^2$&\hspace{3em}$x\mapsto 2^{x}$\\
				7. $f:\mathbb{R}^{2}\rightarrow\mathbb{R}^{3}$&8. $f:\mathbb{R}^{2}\rightarrow\mathbb{R}^{3}$&\\
				\hspace{3em}$\begin{bmatrix}
					x\\y
				\end{bmatrix}\mapsto\begin{bmatrix}
					x+y\\y-x\\2x
				\end{bmatrix}$&\hspace{3em}$\begin{bmatrix}
					x\\y
				\end{bmatrix}\mapsto\begin{bmatrix}
					x+1\\y-x\\2x
				\end{bmatrix}$&
			\end{tabular}
		\end{table}
	\end{exercise}
	\begin{solution}
		\begin{table}[htbp]
			\centering
			\begin{tabular}{p{0.05\textwidth}p{0.3\textwidth}p{0.6\textwidth}}
				\toprule
				序号&满足线性映射的条件?&解释\\
				\midrule
				1. &$\times$ & $f(0)=1\neq 0$\\
				\hline
				2. &$\surd$ & $\begin{aligned}
					f(ax+by)&=2(ax+by)\\
					&=a\cdot(2x)+b\cdot(2y)\\
					&=af(x)+bf(y)
				\end{aligned}$\\
				\hline
				3. &$\surd$ & $\begin{aligned}
					f(ax+by)&=0\\
					&=a\cdot 0+b\cdot 0\\
					&=af(x)+bf(y)
				\end{aligned}$\\
				\hline
				4. & $\times$ & $f(0)=1\neq 0$\\
				\hline
				5. & $\times$ & $4=f(2) \neq 2f(1) = 2$\\
				\hline
				6. & $\times$ & $f(0)=1\neq 0$\\
				\hline 
				7. & $\surd$ & $\begin{aligned}
					f\left(a\begin{bmatrix}
						x_1\\y_1
					\end{bmatrix}+b\begin{bmatrix}
						x_2\\y_2
					\end{bmatrix}\right)&=\begin{bmatrix}
						(ax_1+bx_2)+(ay_1+by_2)\\
						(ay_1+by_2)-(ax_1+bx_2)\\
						2(ax_1+bx_2)
					\end{bmatrix}\\
					&=a\begin{bmatrix}
						x_1+y_1\\
						y_1-x_1\\
						2x_1
					\end{bmatrix}+b\begin{bmatrix}
						x_2+y_2\\
						y_2-x_2\\
						2x_2
					\end{bmatrix}\\
					&=af\left(\begin{bmatrix}
						x_1\\y_1
					\end{bmatrix}\right)+bf\left(\begin{bmatrix}
						x_2\\y_2
					\end{bmatrix}\right)
				\end{aligned}$\\
				\hline
				8. & $\times$ & $f\left(\begin{bmatrix}
					0&0
				\end{bmatrix}^T\right)=\begin{bmatrix}
					1&0&0
				\end{bmatrix}^T\neq\begin{bmatrix}
					0&0&0
				\end{bmatrix}^T$\\
				\bottomrule
			\end{tabular}
		\end{table}
	\end{solution}
	
	\begin{exercise}[1.1.8]
		设 $\mathbf{x_1}=\begin{bmatrix}
			1\\0
		\end{bmatrix}$、$\mathbf{x_2}=\begin{bmatrix}
			1\\1
		\end{bmatrix}$、$\mathbf{x_3}=\begin{bmatrix}
			0\\1
		\end{bmatrix}$、$\mathbf{b_1}=\begin{bmatrix}
			1\\1\\0
		\end{bmatrix}$、$\mathbf{b_2}=\begin{bmatrix}
			1\\0\\1
		\end{bmatrix}$、$\mathbf{b_3}=\begin{bmatrix}
			0\\1\\1
		\end{bmatrix}$,是否存在线性映射 $f: \mathbb{R}^2\rightarrow\mathbb{R}^3$,满足 $f(\mathbf{x_i})=\mathbf{x_i}$,$i=1,2,3$?
	\end{exercise}
	\begin{solution}
		如果存在这样的线性映射,那么会推导出
		\begin{align*}
			\begin{bmatrix}
				1\\0\\1
			\end{bmatrix}=\mathbf{b_2}&=f(\mathbf{x_2})\\
			&=f(\mathbf{x_1}+\mathbf{x_3})\\
			&=f(\mathbf{x_1})+f(\mathbf{x_3})\\
			&=\begin{bmatrix}
				1\\1\\0
			\end{bmatrix}+\begin{bmatrix}
				0\\1\\1
			\end{bmatrix}=\begin{bmatrix}
				1\\2\\1
			\end{bmatrix}
		\end{align*}
		推导出矛盾,所以不存在满足原题的线性映射。
	\end{solution}
\end{document}\


