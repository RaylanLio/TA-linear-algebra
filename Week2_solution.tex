\documentclass[a4paper]{article}
\usepackage{xeCJK}
\usepackage{indentfirst}
\usepackage{amsmath,amssymb,amsthm}
\usepackage{rotating}
\usepackage[margin=2.5cm]{geometry}
\usepackage{booktabs}
\theoremstyle{definition}
\newenvironment{exercise}[1][0.0.0]{\noindent\textbf{习题 #1}\hspace{1em}}{\quad\\}
\newenvironment{solution}[1][]{\noindent\textbf{\textit{解:}}}{\quad\\}
\newenvironment{case}[2]{\noindent\textbf{情况 #1:}#2\\}{\quad\\}
\newenvironment{step}[2]{\noindent\textbf{步骤 #1:}#2\\}{\quad\\}
\linespread{1.4}

\author{刘思齐老师;廖汶锋、乐永琪、杜若甫助教}

\title{\textbf{线性代数——第二周作业解答}}
\begin{document}
	\maketitle
	
	\iffalse week 2
	练习 1.1.10
	练习 1.1.11
	练习 1.2.1
	练习 1.2.2(6,7,8,9,10)
	练习 1.2.3
	练习 1.2.5(2,3,4,5)
	练习 1.2.7
	\fi
	
	\begin{exercise}[1.1.10]
		给定三维向量 $\boldsymbol{a}=\begin{bmatrix}
			a_1\\a_2\\a_3
		\end{bmatrix}$,$\boldsymbol{b}=\begin{bmatrix}
			b_1\\b_2\\b_3
		\end{bmatrix}$,定义
		\begin{enumerate}
			\item 二者\textbf{点积}为 $\boldsymbol{a\cdot b}:=a_1b_1+a_2b_2+a_3b_3$
			\item 二者\textbf{叉积}为 $\boldsymbol{a\times b}:=\begin{bmatrix}
				a_2b_3-a_3b_2\\
				a_3b_1-a_1b_3\\
				a_1b_2-a_2b_1
			\end{bmatrix}$
		\end{enumerate}
		那么给定 $\boldsymbol{a}\in\mathbb{R}^3$,映射 $f:\mathbb{R}^3\rightarrow\mathbb{R}$,$\boldsymbol{b}\mapsto\boldsymbol{a\cdot b}$ 和 $g:\mathbb{R}^3\rightarrow\mathbb{R}^3$,$\boldsymbol{b}\mapsto\boldsymbol{a\times b}$ 是否线性映射?
	\end{exercise}
	\begin{solution}
		对于任意 $x$、$y\in\mathbb{R}$ 和 $\boldsymbol{b}=\begin{bmatrix}
			b_1\\b_2\\b_3
		\end{bmatrix}$、$\boldsymbol{c}=\begin{bmatrix}
			c_1\\c_2\\c_3
		\end{bmatrix}\in\mathbb{R}^3$ 来说
		\begin{enumerate}
			\item 点积映射
			\begin{align*}
				f(x\boldsymbol{b}+y\boldsymbol{c})&=a_1(xb_1+yc_1)+a_2(xb_2+yc_2)+a_3(xb_3+yc_3)\\
				&=x(a_1b_1+a_2b_2+a_3b_3)+y(a_1c_1+a_2c_2+a_3c_3)\\
				&=xf(\boldsymbol{b})+yf(\boldsymbol{c})
			\end{align*}
			所以 $f$ 是线性映射。
			\item 叉积映射
			\begin{align*}
				g(x\boldsymbol{b}+y\boldsymbol{c})&=\begin{bmatrix}
					a_2(xb_3+yc_3)-a_3(xb_2+yc_2)\\
					a_3(xb_1+yc_1)-a_1(xb_3+yc_3)\\
					a_1(xb_2+yc_2)-a_2(xb_1+yc_1)
				\end{bmatrix}\\
				&=x\begin{bmatrix}
					a_2b_3-a_3b_2\\
					a_3b_1-a_1b_3\\
					a_1b_2-a_2b_1
				\end{bmatrix}+y\begin{bmatrix}
				a_2c_3-a_3c_2\\
				a_3c_1-a_1c_3\\
				a_1c_2-a_2c_1
				\end{bmatrix}\\
				&=xg(\boldsymbol{b})+yg(\boldsymbol{c})
			\end{align*}
			所以 $g$ 也是线性映射。
		\end{enumerate}
	\end{solution}
	 \newpage
	\begin{exercise}[1.1.11]
		给定平面上任意面积为 1 的三角形,经过下列变换之后,其面积是否确定?如果是,面
		积为多少?(不需严谨证明,猜测答案即可。)
		\begin{table}[htbp]
			\centering
			\begin{tabular}{p{0.45\textwidth}p{0.45\textwidth}}
				1. 旋转变换;&2. 反射变换;\\
				3. 对 $x_2$ 投影的投影变换;&4. $x_1$ 方向拉伸 3 倍,$x_2$ 方向不变的伸缩变换;\\
				5. 把 $x_1$ 方向的 3 倍加到 $x_2$ 方向上,保持 $x_1$ 方向不变的错切变换。
			\end{tabular}
		\end{table}
	\end{exercise}
	\begin{solution}
		对于旋转变换、反射变换和投影变换三者,三角形的面积显然变成 1、1 和 0;对于伸缩变换和错切变换,三角形面积变成了 3 和 1,在附录将给出证明,作为选读材料。
	\end{solution}
	
	\begin{exercise}[1.2.1]
		将下列向量 $\boldsymbol{b}$ 写成矩阵和向量乘积的形式:
		\begin{table}[htbp]
			\centering
			\begin{tabular}{p{0.45\textwidth}p{0.45\textwidth}}
				1. $\boldsymbol{b}=2\begin{bmatrix}
					1\\1
				\end{bmatrix}+4\begin{bmatrix}
					0\\1
				\end{bmatrix}+5\begin{bmatrix}
					1\\0
				\end{bmatrix}$; & 2. $\boldsymbol{b}=5\begin{bmatrix}
					1\\2\\3\\4\\5
				\end{bmatrix}+4\begin{bmatrix}
					5\\4\\3\\2\\1
				\end{bmatrix}$;\\
				3. $\boldsymbol{b}=\begin{bmatrix}
					2b+a+c\\c-b\\a+b+c\\a+b
				\end{bmatrix}$;&4. $\boldsymbol{b}=f\left(\begin{bmatrix}
					1\\2\\3\\4\\5
				\end{bmatrix}\right)$,其中 $f:\mathbb{R}^5\rightarrow\mathbb{R}^5$ 是线性变换,满足 $f(\boldsymbol{e_k})=k\boldsymbol{e_{6-k}}$,$k=1,2,\cdot,5$;
			\end{tabular}
		\end{table}
		
		5. 假设如果某天下雨,则第二天下雨概率为 0.8;如果当天不下雨,则第二天下雨概率为 0.3。
		
		已知当天有一半的概率会下雨,令 $\boldsymbol{b}\in\mathbb{R}^2$,其两个分量分别是明天下雨和不下雨的概率。
	\end{exercise}
	
	\begin{solution}
		\begin{table}[htbp]
			\centering
			\begin{tabular}{p{0.45\textwidth}p{0.45\textwidth}}
				1. $\boldsymbol{b}=\begin{bmatrix}
					1&0&1\\
					1&1&0
				\end{bmatrix}\begin{bmatrix}
					2\\4\\5
				\end{bmatrix}$ & 2. $\boldsymbol{b}=\begin{bmatrix}
				1&5\\2&4\\3&3\\4&2\\5&1
				\end{bmatrix}\begin{bmatrix}
				5\\4
				\end{bmatrix}$\\
				3.  $\boldsymbol{b}=\begin{bmatrix}
					1&2&1\\0&-1&1\\1&1&1\\1&1&0
				\end{bmatrix}\begin{bmatrix}
					a\\b\\c
				\end{bmatrix}$ & 4.  $\boldsymbol{b}=\begin{bmatrix}
				0&0&0&0&5\\
				0&0&0&4&0\\
				0&0&3&0&0\\
				0&2&0&0&0\\
				1&0&0&0&0\\
				\end{bmatrix}\begin{bmatrix}
				1\\2\\3\\4\\5
				\end{bmatrix}$\\
				5. $\boldsymbol{b}=\begin{bmatrix}
					0.8&0.3\\
					0.2&0.7
				\end{bmatrix}\begin{bmatrix}
					0.5\\0.5
				\end{bmatrix}$
			\end{tabular}
		\end{table}
	\end{solution}
	
	\begin{exercise}[1.2.2]
		判断下列矩阵和向量的乘积是否良定义。在可以计算时,先将其写成列向量的线性组合,再进行计算:
		\begin{table}[htbp]
			\centering
			\begin{tabular}{p{0.33\textwidth}p{0.33\textwidth}p{0.33\textwidth}}
				6. $\begin{bmatrix}
					1&2&3\\
					4&5&6\\
					7&8&9
				\end{bmatrix}\begin{bmatrix}
					1\\-2\\1
			\end{bmatrix}$;&7. $\begin{bmatrix}
				1&4&7\\
				2&5&8\\
				3&6&9
			\end{bmatrix}\begin{bmatrix}
				1\\-2\\1
			\end{bmatrix}$;&8. $\begin{bmatrix}
				1&1&0\\
				3&2&1\\
				7&4&3
			\end{bmatrix}\begin{bmatrix}
				1\\-1\\-1
			\end{bmatrix}$;\\
			9. $\begin{bmatrix}
				1&3&7\\
				1&2&4\\
				0&1&3
			\end{bmatrix}\begin{bmatrix}
				1\\-1\\-1
			\end{bmatrix}$;&10. $\begin{bmatrix}
			1&3&7\\
			1&2&4\\
			0&1&3
			\end{bmatrix}\begin{bmatrix}
			2\\-3\\-1
			\end{bmatrix}$。
			\end{tabular}
		\end{table}
	\end{exercise}
	\begin{solution}
		\begin{enumerate}
			\item[6.] $\begin{bmatrix}
				1&2&3\\
				4&5&6\\
				7&8&9
			\end{bmatrix}\begin{bmatrix}
				1\\-2\\1
			\end{bmatrix}=1\begin{bmatrix}
				1\\4\\7
			\end{bmatrix}-2\begin{bmatrix}
				2\\5\\8
			\end{bmatrix}+1\begin{bmatrix}
				3\\6\\9
			\end{bmatrix}=\begin{bmatrix}
				0\\0\\0
			\end{bmatrix}$
			\item[7.] $\begin{bmatrix}
				1&4&7\\
				2&5&8\\
				3&6&9
			\end{bmatrix}\begin{bmatrix}
				1\\-2\\1
			\end{bmatrix}=1\begin{bmatrix}
				1\\2\\3
			\end{bmatrix}-2\begin{bmatrix}
				4\\5\\6
			\end{bmatrix}+1\begin{bmatrix}
				7\\8\\9
			\end{bmatrix}=\begin{bmatrix}
				0\\0\\0
			\end{bmatrix}$
			\item[8.] $\begin{bmatrix}
				1&1&0\\
				3&2&1\\
				7&4&3
			\end{bmatrix}\begin{bmatrix}
				1\\-1\\-1
			\end{bmatrix}=1\begin{bmatrix}
				1\\3\\7
			\end{bmatrix}-1\begin{bmatrix}
				1\\2\\4
			\end{bmatrix}-1\begin{bmatrix}
				0\\1\\3
			\end{bmatrix}=\begin{bmatrix}
				0\\0\\0
			\end{bmatrix}$
			\item[9.] $\begin{bmatrix}
				1&3&7\\
				1&2&4\\
				0&1&3
			\end{bmatrix}\begin{bmatrix}
				1\\-1\\-1
			\end{bmatrix}=1\begin{bmatrix}
				1\\1\\0
			\end{bmatrix}-1\begin{bmatrix}
				3\\2\\1
			\end{bmatrix}-1\begin{bmatrix}
				7\\4\\3
			\end{bmatrix}=\begin{bmatrix}
				-9\\-5\\-4
			\end{bmatrix}$
			\item[10.] $\begin{bmatrix}
				1&3&7\\
				1&2&4\\
				0&1&3
			\end{bmatrix}\begin{bmatrix}
				2\\-3\\-1
			\end{bmatrix}=2\begin{bmatrix}
				1\\1\\0
			\end{bmatrix}-3\begin{bmatrix}
				3\\2\\1
			\end{bmatrix}+1\begin{bmatrix}
				7\\4\\3
			\end{bmatrix}=\begin{bmatrix}
				0\\0\\0
			\end{bmatrix}$
		\end{enumerate}
	\end{solution}
	
	\newpage
	\begin{exercise}[1.2.3]
		设 $A= \begin{bmatrix}
			0.8&0.3\\0.2&0.7
		\end{bmatrix}$ ,$\boldsymbol{u_0}=\begin{bmatrix}
			1\\0
		\end{bmatrix}$,$\boldsymbol{v_0}=\begin{bmatrix}
			1\\3
		\end{bmatrix}$。对任意自然数 $i$,令 $\boldsymbol{u_{i+1}}=A\boldsymbol{u_i}$,$\boldsymbol{v_{i+1}}=A\boldsymbol{v_i}$。
		\begin{enumerate}
			\item 对$i=1,2,3,4$,计算 $u_i$,$v_i$;
			\item 猜测 $\lim\limits_{i\rightarrow\infty} \boldsymbol{u_i}$,$\lim\limits_{i\rightarrow\infty}\boldsymbol{v_i}$;
			\item 任取初始向量 $\boldsymbol{w_0}$,猜测极限 $\lim\limits_{i\rightarrow\infty}\boldsymbol{w_i}$,其中 $\boldsymbol{w_{i+1}}=A\boldsymbol{w_i}$;不同初始向量得到的极限在同一
			条直线上吗?
		\end{enumerate}
	\end{exercise}
	\begin{solution}
		\begin{enumerate}
			\item 计算 $\boldsymbol{u_i}$ 和 $\boldsymbol{v_i}$
			\begin{table}[htbp]
				\centering
				\begin{tabular}{p{0.24\textwidth}p{0.24\textwidth}p{0.24\textwidth}p{0.24\textwidth}}
					$\begin{aligned}
						\boldsymbol{u_1}&= \begin{bmatrix}
							0.8&0.3\\0.2&0.7
						\end{bmatrix}\begin{bmatrix}
							1\\0
						\end{bmatrix}\\
						&=\begin{bmatrix}
							0.8\\0.2
						\end{bmatrix}
					\end{aligned}$&$\begin{aligned}
						\boldsymbol{u_2}&= \begin{bmatrix}
							0.8&0.3\\0.2&0.7
						\end{bmatrix}\begin{bmatrix}
							0.8\\0.2
						\end{bmatrix}\\
						&=\begin{bmatrix}
							0.7\\0.3
						\end{bmatrix}
					\end{aligned}$&$\begin{aligned}
						\boldsymbol{u_3}&= \begin{bmatrix}
							0.8&0.3\\0.2&0.7
						\end{bmatrix}\begin{bmatrix}
							0.7\\0.3
						\end{bmatrix}\\
						&=\begin{bmatrix}
							0.65\\0.35
						\end{bmatrix}
					\end{aligned}$&$\begin{aligned}
						\boldsymbol{u_4}&= \begin{bmatrix}
							0.8&0.3\\0.2&0.7
						\end{bmatrix}\begin{bmatrix}
							0.65\\0.35
						\end{bmatrix}\\&=\begin{bmatrix}
							0.625\\0.375
						\end{bmatrix}
					\end{aligned}$\\
					$\begin{aligned}
						\boldsymbol{v_1}&= \begin{bmatrix}
							0.8&0.3\\0.2&0.7
						\end{bmatrix}\begin{bmatrix}
							1\\3
						\end{bmatrix}\\
						&=\begin{bmatrix}
							1.7\\2.3
						\end{bmatrix}
					\end{aligned}$&$\begin{aligned}
						\boldsymbol{v_2}&= \begin{bmatrix}
							0.8&0.3\\0.2&0.7
						\end{bmatrix}\begin{bmatrix}
							1.7\\2.3
						\end{bmatrix}\\
						&=\begin{bmatrix}
							2.05\\1.95
						\end{bmatrix}
					\end{aligned}$&$\begin{aligned}
						\boldsymbol{v_3}&= \begin{bmatrix}
							0.8&0.3\\0.2&0.7
						\end{bmatrix}\begin{bmatrix}
							2.05\\1.95
						\end{bmatrix}\\
						&=\begin{bmatrix}
							2.225\\1.775
						\end{bmatrix}
					\end{aligned}$&$\begin{aligned}
						\boldsymbol{v_4}&= \begin{bmatrix}
							0.8&0.3\\0.2&0.7
						\end{bmatrix}\begin{bmatrix}
							2.225\\1.775
						\end{bmatrix}\\&=\begin{bmatrix}
							2.3125\\1.6875
						\end{bmatrix}
					\end{aligned}$
				\end{tabular}
			\end{table}
			\item 猜测 $\lim\limits_{i\rightarrow\infty} \boldsymbol{u_i}$,$\lim\limits_{i\rightarrow\infty}\boldsymbol{v_i}$:
			
			自行计算 $i=5,6,7,8$ 的结果,会发现 $\boldsymbol{u_i}$ 趋向于 $\begin{bmatrix}
				0.6\\0.4
			\end{bmatrix}$,而 $\boldsymbol{v_i}$ 趋向于 $\begin{bmatrix}
				2.4\\1.6
			\end{bmatrix}$
			
			因此猜测 $\lim\limits_{i\rightarrow\infty} \boldsymbol{u_i}=\begin{bmatrix}
				0.6\\0.4
			\end{bmatrix}$,$\lim\limits_{i\rightarrow\infty}\boldsymbol{v_i}=\begin{bmatrix}
				2.4\\1.6
		\end{bmatrix}$。
		\item 猜测 $\lim\limits_{i\rightarrow\infty}\boldsymbol{w_i}$:
		
		根据2. 的推测,发现第一行元素和第二行元素之比趋近 $3:2$,且两元素的总和不变,所以对于任意 $w_0$ 来说,
		\[\lim\limits_{i\rightarrow\infty}\boldsymbol{w_0}=\left(\begin{bmatrix}
			1&1
		\end{bmatrix}\boldsymbol{w_0}\right)\begin{bmatrix}
			0.6\\0.4
		\end{bmatrix}\]
		并且所有向量的极限都在 $y=\frac{2}{3}x$ 的直线上。
		\end{enumerate}
		
		(*对于 $\lim\limits_{i\rightarrow\infty}\boldsymbol{w_0}$ 的详细计算,笔者会放在附录中,同学们选读)
	\end{solution}
	
	\newpage
	\begin{exercise}[1.2.5]
		计算下列线性映射 $f$ 的表示矩阵:
		\begin{enumerate}
			\item[2.] $f:\mathbb{R}^3\rightarrow\mathbb{R}$,$f(\boldsymbol{x})=\begin{bmatrix}
				1\\2\\3
			\end{bmatrix}\cdot\boldsymbol{x}$,其中点积的定义见习题 1.1.10;
			\item[3.] $f:\mathbb{R}^3\rightarrow\mathbb{R}$,$f(\boldsymbol{x})=\begin{bmatrix}
				1\\2\\3
			\end{bmatrix}\times\boldsymbol{x}$,其中叉积的定义见习题 1.1.11;
			\item[4.] $f:\mathbb{R}^4\rightarrow\mathbb{R}^4$,$f\left(\begin{bmatrix}
				x_1\\x_2\\x_3\\x_4
			\end{bmatrix}\right)=\begin{bmatrix}
				x_1\\x_2\\x_3\\x_4+x_2
			\end{bmatrix}$
			\item[5.] $f:\mathbb{R}^4\rightarrow\mathbb{R}^4$,$f\left(\begin{bmatrix}
				x_1\\x_2\\x_3\\x_4
			\end{bmatrix}\right)=\begin{bmatrix}
				x_4\\x_2\\x_3\\x_1
			\end{bmatrix}$
	
		\end{enumerate}
	\end{exercise}
	\begin{solution} 记表示矩阵为 $A$
		\begin{table}[htbp]
			\centering
			\begin{tabular}{p{0.3\textwidth}p{0.3\textwidth}p{0.3\textwidth}}
				2. $A=\begin{bmatrix}
				1&2&3
				\end{bmatrix}$&3. $A=\begin{bmatrix}
				0&-3&2\\
				3&0&-1\\
				-2&1&0
				\end{bmatrix}$\\
				4. $A=\begin{bmatrix}
					1&0&0&0\\
					0&1&0&0\\
					0&0&1&0\\
					0&1&0&1
				\end{bmatrix}$& 5. $A=\begin{bmatrix}
				0&0&0&1\\
				0&1&0&0\\
				0&0&1&0\\
				1&0&0&0
				\end{bmatrix}$ 
			\end{tabular}
		\end{table}
	\end{solution}
	
	\begin{exercise}[1.2.7]
		设 $xy$ 平面 $\mathbb{R}^2$ 上的变换 $f$ 是下列三个变换的复合:先绕原点逆时针旋转 $\frac{\pi}{6}$;然后进行一个保持 $y$ 坐标不变,同时将 $y$ 坐标的两倍加到 $x$ 坐标上的错切;最后再沿着直线 $x+y=0$ 反射。
		\begin{table}[htbp]
			\centering
			\begin{tabular}{p{0.45\textwidth}p{0.45\textwidth}}
				1. 证明 $f$ 是线性变换;&2. 计算 $f(\boldsymbol{e_1})$ 和 $f(\boldsymbol{e_2})$;\\
				3. 写出 $f$ 的表示矩阵;&4. 计算 $f\left(\begin{bmatrix}
					3\\4
				\end{bmatrix}\right)$。
			\end{tabular}
		\end{table}
	\end{exercise}
	\begin{solution}
		\begin{enumerate}
			\item 证明 $f$ 是线性变换:
			
			先证明以下引理:考虑任意两个线性变换 $g:\mathbb{R}^m\rightarrow\mathbb{R}^m$ 和 $h:\mathbb{R}^m\rightarrow\mathbb{R}^m$,任取 $\boldsymbol{x}$、$\boldsymbol{y}\in\mathbb{R}^m$ 和 $a$、$b\in\mathbb{R}$,其复合变换 $g\circ h$ 也是线性变换。
			
			证明:\begin{align*}
				g\circ h(a\boldsymbol{x}+b\boldsymbol{y})&=g(ah(\boldsymbol{x})+bh(\boldsymbol{y}))\\
				&=ag(h(\boldsymbol{x}))+bg(h(\boldsymbol{y}))\\
				&=ag\circ h(\boldsymbol{x}) + bg\circ h(\boldsymbol{y})
			\end{align*}
			证毕
			
			因为旋转变换、错切变换和反射变换都是线性变换,结合引理可知,三者的复合变换必定是线性变换。
			
			\item 计算 $f(\boldsymbol{e_1})$ 和 $f(\boldsymbol{e_2})$:
			\[
				\boldsymbol{e_1}\stackrel{\text{逆时针旋转}\frac{\pi}{6}}{\longmapsto}\begin{bmatrix}
					\frac{\sqrt{3}}{2}\\\frac{1}{2}
				\end{bmatrix}\stackrel{\text{错切}}{\longmapsto}\begin{bmatrix}
				\frac{\sqrt{3}}{2}+1\\\frac{1}{2}
				\end{bmatrix}\stackrel{\text{沿着直线} x+y=0 \text{反射}}{\longmapsto}\begin{bmatrix}
				-\frac{1}{2}\\-\frac{\sqrt{3}}{2}-1
				\end{bmatrix}=f(\boldsymbol{e_1})
			\]
			\[
					\boldsymbol{e_2}\stackrel{\text{逆时针旋转}\frac{\pi}{6}}{\longmapsto}\begin{bmatrix}
						-\frac{1}{2}\\\frac{\sqrt{3}}{2}
					\end{bmatrix}\stackrel{\text{错切}}{\longmapsto}\begin{bmatrix}
						\sqrt{3}-\frac{1}{2}\\\frac{\sqrt{3}}{2}
					\end{bmatrix}\stackrel{\text{沿着直线} x+y=0 \text{反射}}{\longmapsto}\begin{bmatrix}
						-\frac{\sqrt{3}}{2}\\\frac{1}{2}-\sqrt{3}
					\end{bmatrix}=f(\boldsymbol{e_2})
			\]
			
			\item 写出 $f$ 的表示矩阵$A$:
			\[
				A=\begin{bmatrix}
					f(\boldsymbol{e_1})&f(\boldsymbol{e_2})
				\end{bmatrix}=\begin{bmatrix}
					-\frac{1}{2}&-\frac{\sqrt{3}}{2}\\
					-\frac{\sqrt{3}}{2}-1&\frac{1}{2}-\sqrt{3}
				\end{bmatrix}
			\]
			
			\item  计算 $f\left(\begin{bmatrix}
				3\\4
			\end{bmatrix}\right)$:
			
			\begin{align*}
				f\left(\begin{bmatrix}
					3\\4
				\end{bmatrix}\right)&=A\begin{bmatrix}
					3\\4
				\end{bmatrix}\\
				&=\begin{bmatrix}
					-2\sqrt{3}-\frac{3}{2}\\
					-\frac{11}{2}\sqrt{3}-1
				\end{bmatrix}
			\end{align*}
		\end{enumerate}
	\end{solution}
	
	\newpage
	\section*{附录}
	\textbf{习题1.1.11} 对顶点坐标为 $(x_i,y_i)$,$i=1,2,3$ 的三角形来说,其面积 $S$ 是
	\[
		S=\frac{1}{2}\left|\det\begin{bmatrix}
			1&1&1\\
			x_1&x_2&x_3\\
			y_1&y_2&y_3
		\end{bmatrix}\right|=1
	\]
	
	对于题中的伸缩变换来说,变换后的面积是(矩阵之积的行列式等于矩阵行列式之积)
	\[
		S'=\frac{1}{2}\left|\det\begin{bmatrix}
			1&1&1\\
			3x_1&3x_2&3x_3\\
			y_1&y_2&y_3
		\end{bmatrix}\right|=\frac{1}{2}\left|\det\begin{bmatrix}
			1&0&0\\
			0&3&0\\
			0&0&1
		\end{bmatrix}\det\begin{bmatrix}
		1&1&1\\
		x_1&x_2&x_3\\
		y_1&y_2&y_3
		\end{bmatrix}\right|=\left|\det\begin{bmatrix}
		1&0&0\\
		0&3&0\\
		0&0&1
		\end{bmatrix}\right|S=3
	\]
	
	而对于错切变换来说
	\[
		S'=\frac{1}{2}\left|\det\begin{bmatrix}
			1&1&1\\
			x_1&x_2&x_3\\
			3x_1+y_1&3x_2+y_2&3x_3+y_3
		\end{bmatrix}\right|=\frac{1}{2}\left|\det\begin{bmatrix}
		1&1&1\\
		x_1&x_2&x_3\\
		y_1&y_2&y_3
		\end{bmatrix}\right|=1
	\]
	\quad\\[1em]
	
	\textbf{习题1.2.3} 考虑任意 $\boldsymbol{w_0}=\begin{bmatrix}
		w_{01}\\w_{02}
	\end{bmatrix}$,并记$\boldsymbol{w_n}=\begin{bmatrix}
		w_{n1}\\w_{n2}
	\end{bmatrix}$。
	
	首先注意
	\[
		w_{(n+1)1}+w_{(n+1)2}=\begin{bmatrix}
			1&1
		\end{bmatrix}\boldsymbol{w_{n+1}}=\begin{bmatrix}
		1&1
		\end{bmatrix}A\boldsymbol{w_{n}}=\begin{bmatrix}
		1&1
		\end{bmatrix}\boldsymbol{w_{n}}=w_{n1}+w_{n2}
	\]
	
	所以可以记 $S=w_{01}+w_{02}$,$\boldsymbol{w_n}=\begin{bmatrix}
		w_{n1}\\S-w_{n1}
	\end{bmatrix}$。
	
	然后就可以考察 $w_{n1}$ 的递推公式
	\begin{align*}
		&w_{(n+1)1}=0.5w_{n1}+0.3S\\
		\Leftrightarrow& w_{(n+1)1}-0.6S = 0.5(w_{n1}-0.6S)\\
		\Leftrightarrow& w_{n1}=0.6S+0.5^{n-1}(w_{01}-0.6S)
	\end{align*}
	
	所以
	\[
		\begin{bmatrix}
			\boldsymbol{w_n}=\begin{bmatrix}
				0.6S+0.5^{n-1}(w_{01}-0.6S)\\
				0.4S-0.5^{n-1}(w_{01}-0.6S)
			\end{bmatrix}
		\end{bmatrix}
	\]

	当 $n\rightarrow \infty$ 时,$\boldsymbol{w_n}\rightarrow\begin{bmatrix}
		0.6S\\0.4S
	\end{bmatrix}$,所以 $\lim\limits_{n\rightarrow\infty}\boldsymbol{w_n}=\left(\begin{bmatrix}
	1&1
	\end{bmatrix}\boldsymbol{w_0}\right)\begin{bmatrix}
		0.6\\0.4
	\end{bmatrix}$。
\end{document}\


